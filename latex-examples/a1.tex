\documentclass{article}
\usepackage{amsmath}
\title{Assignment 1}
\author{Ross Weir --- 20282025}
\date{}

\begin{document}
\maketitle
\begin{enumerate}
	\item{\em What is the formula for compound interest?}\\
		$$P' = P(1+\frac{r}{n})^{nt}$$

		Where:
		\par$P$ is the starting balance
		\par$P'$ is the resulting balance
		\par$r$ is the interest rate for a given period of time
		\par$n$ is the number of compounding intervals per rate
		\par$t$ is the number of time periods total.\\

		Ex: if r is 5\% annually, then n would be the number of compounding periods within a year (such as one per month = 12), and t would be the number of years.

		As such, a $P=\$1000$ loan at $r=5\%$ per annum, with $n=12$ compounding periods and no loan payments for $t=4$ years would have a balance of:
		
		\begin{align*}
		P'&= 1000\times(1+\frac{0.05}{12}) ^ {12\times4} \\
		P'&=\$1220.90
		\end{align*}

	\item{\em What is the derivative of $e^x$?}\\
		$$\frac{d}{dx} e^x = e^x$$
\end{enumerate}
\end{document}